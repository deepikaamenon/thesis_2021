\chapter{Results}    \label{results}
\section{Recruitment of Rvs to endocytic sites}

	\subparagraph{Curvature sensing versus generation }
	\mbox{}\\
Cellular membrane shape is a result of properties like rigidity, tension, intracellular pressure, and are influenced by the lipid composition and the proteins embedded in it1,2. Since tension, pressure, and rigidity all oppose membrane deformation, energy is required to deform and bend it. BAR domains can generate curvature if the energy required to deform the membrane is less than the energy spent in binding flat membrane.

\vspace{5mm}
			
Scaffolding as a curvature-generation mechanism has been extended to various types of BAR proteins, (Arkhipov et al., 2009; Frost et al., 2008; Henne et al., 2007; Itoh et al., 2005; Pykalainen et al., 2011; Saarikangas et al., 2009; Shimada et al., 2007; Yu and Schulten, 2013). In order for BAR scaffolds to impose membrane curvature, some requirement have to be met3: they have to have present a large membrane-interacting surface that can mediate membrane binding, have intrinsic curvature that can be imposed on the surface, and have a rigid structure that can overcome bending resistance of the membrane. Because of their shape (Peter 2004, Gallop 2006, Weissenhorn 2005), and their capacity to oligomerize into large assemblies on tubes (Mim 2012, Mizuno 2010, Takei 1999, Yin 2009), it has been suggested that BAR domains impose their shape on the membrane, and generate membrane curvature in cells. Further, tubulation both in-vivo and of liposomes is dependent on the rigidity of the central crescent-shaped domain4. The N-helix of NBAR domains can generate curvature independently of the BAR scaffold (Varkey 2010, Westphal and Chandra 2013).

\vspace{5mm}
For endophilin, the BAR domain is relatively far from the membrane, suggesting a mechanism dependent on the N-helix (Jao 2010). Different BAR domains thus likely employ different mechanisms to interact with the membrane for generating vesicles, and tubes (Ambroso 2014). For endophilin, for example, the N-helix is necessary for liposome binding5, while that of amphiphysin is important, but not necessary6. 

\vspace{5mm}
Curved BAR proteins that can induce curvature are also able to sense curvature: in-vitro, BAR domains show a preferential-binding to vesicles based on their intrinsic curvature. Curvature-generation and sensing seem to intrinsically couple mechanisms. That BAR domains are able to generate curvature does not imply that this is their function, at least in endocytocis: in-vivo, the significance of curvature-generation is not determined. Tracking over thirty different endocytic proteins in NIH-3TC cells (derived from mouse fibroblasts), TIRF imaging shows that Endophilin2 and Amphiphysin1 arrive late in the endocytic time-line right before scission7, suggesting they arrive when membrane tubes are already formed. 

\vspace{5mm}
In the case of Rvs, centroid tracking and averaging shows that the complex similarly localizes to sites late in the endocytic timeline, close to scission8. CLEM studies have further shown that Rvs localizes to sites after the membrane invaginations are about 60nm deep into the cytoplasm: Rvs localizes once membrane curvature is established. Whether this localization is dependent on membrane curvature, recognized by the BAR domain has not been shown. 

\vspace{5mm}
Rvs localization in yeast endocytosis
As has been shown before, Rvs localizes to endocytic patches at the yeast plasma membrane in the late scission-stage. When imaged at the plasma membrane, rather than a slow inward movement typical for coat proteins like Sla1, Rvs167-GFP shows a sharp jump into the cytoplasm, concomitant with scission8,9. The average lifetime of Rvs167-GFP is about 7secs, as measured by epifluorescence microscopy at the equatorial plane of a haploid yeast cell. 


	\subsection{BAR domain senses membrane curvature in-vivo}
	\subsection{SH3 domain is responsible for actin-independent	\\
		localization of Rvs}
	\subsection{What does the SH3 domain interact with?}
		\subsubsection{Type 1 myosins}
		\subsubsection{Las17}
		\subsubsection{Vrp1}

	\subsection{Other potential mechanisms of assembly/ disassembly of Rvs}		
			\subsubsection{Interaction with Calmodulin}
			\subsubsection{FBAR protein Bzz1}
				
\section{Role of the SH3 domain}	
Is likely that SH3 domains, are involved in modulating oligomerization (5, 14) and MC-S, G (15, 82). 	
\section{Role of the N-helix}			
		
\section{Scission mechanisms}

	\subsection{Membrane scission is not dependent on Vps1}
	\subsection{Yeast Synaptojanins do not influence scission timing}
	\subsection{Protein friction does not induce membrane scission }
	\subsection{Scission timing is determined by actin forces, \\
		BAR domains prevent scission}