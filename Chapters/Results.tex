\chapter{Results}    \label{results}
\section{Recruitment of Rvs to endocytic sites}

	\subparagraph{Curvature sensing or generation? }
	\mbox{}\\
	This is where the results go. And then some more results. And more.
In addition, some BAR proteins act as effectors or regula- tors of Rho-family GTPases: they recruit the phosphoinositide phosphatase Synaptojanin, which regulates PI(4,5) P2
and PI(3,4,5)P3 , and dynamin, whose
role in membrane fi ssion is intercon- nected with actin dynamics (Itoh and De Camilli, 2006; Takenawa and Suetsugu,
2007). Other
(Takei et al., 1999; Farsad et al., 2001; Yarar et al., 2007). The

	\subsection{BAR domain senses membrane curvature in-vivo}
	\subsection{SH3 domain is responsible for actin-independent	\\
		localization of Rvs}
	\subsection{What does the SH3 domain interact with?}
		\subsubsection{Type 1 myosins}
		\subsubsection{Las17}
		\subsubsection{Vrp1}

	\subsection{Other potential mechanisms of assembly/ disassembly of Rvs}		
			\subsubsection{Interaction with Calmodulin}
			\subsubsection{FBAR protein Bzz1}
				
\section{Role of the SH3 domain}		
\section{Role of the N-helix}			
		
\section{Scission mechanisms}

	\subsection{Membrane scission is not dependent on Vps1}
	\subsection{Yeast Synaptojanins do not influence scission timing}
	\subsection{Protein friction does not determine yeast scission }
	\subsection{Scission timing is determined by actin forces, while BAR domains prevent scission}