% Chapter Template

\chapter{Aims of the study} % Main chapter title

\label{Ch:Aims} % Change X to a consecutive number; for referencing this chapter elsewhere, use \ref{ChapterX}

The final stage of endocytosis involves breaking a tubular membrane invagination into a vesicle. Over fifty different proteins are involved in establishing endocytic sites, and forming an actin netwrok that pulls up the membrane. Only two proteins are  definitely invovled in the scission stage, Rvs167 and Rvs161 that form the Rvs complex. Rvs arrives at invaginated membranes. Failure to recruit Rvs results in a high rate of scission failure, and smaller vesicles, indicated that Rvs prevents premature membrane scission. In this thesis I 

 \textit{in situ} within the endocytic machinery. More specifically, I addressed the following questions:

\begin{itemize}
	\item How is Rvs recruited to endocytic sites? 
\end{itemize}

The recently developed technique of localization microscopy critically depends on dense and specific fluorescent labeling of the cellular structure of interest with a dye suitable for localization microscopy. In the first part of my project, I established an optimized sample preparation pipeline to enable high quality dual-color localization microscopy of yeast cells. These efforts are described in section \ref{LocMic_yeast}.

\begin{itemize}
	\item What is the function of Rvs?
\end{itemize}