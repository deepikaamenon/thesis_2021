
\chapter{Discussion}    
\label{Ch:discussion}
The recruitment and function of the Rvs complex in the last scission stage of endocytosis has been explored in this work, as well as some previously untested proposals for how membrane scission could be effected in yeast endocytosis. 
I propse that Rvs localizes by interactions of the BAR domains of the Rvs complex with invaginated membranes, and that the SH3 domain is required for efficient and timely recruitment of Rvs is to sites. Arrival of Rvs on membrane tubes then scaffolds the membrane tube and prevents membrane scission, in a manner that depends on recruitment of a critical number of Rvs molecules, till actin forces rupture the membrane, causing vesicle scission. 

Here I discuss the main findings of this thesis in support these propositions.

\section{Recruitment of Rvs to endocytic sites}

\subsection{The BAR- membrane curvature interaction}

It has been shown before that Rvs localizes to already invaginated membrane tubes, suggesting that its role in the last stage is not to drive or significantly influence membrane curvature. From its expected curved structure, it has so far been assumed that it is recruited by its preference for some membrane shapes over others. In the absence of membrane curvature, in sla2del cells, the Rvs BAR domain does not localize to cortical patches. This demonstrate for the first time that the Rvs BAR domain does indeed require membrane curvature to localize to cortical patches. Work on other BAR domains have so far not discovered a specific interaction with a lipid subtype, and have suggested that generic hydrophobic interactions are involved. It is not so far clear how the Rvs BAR domains interacts with the membrane, and mutations of lipid interacting surfaces will be necessary to determine the type of interaction with the underlying lipids. It is also unclear how Rvs is arranged on the membrane tube. Although solved structures of BAR domains show high structural similarity in spite of low sequence similarity, no structure for the Rvs complex exits. The fact that this is a hetero- rather than homodimer suggests that the structure does not necessarily resemble that of Amphiphysin or Endophilin homodimers, and a high-resolution structure will be necessary to clarify the interaction and arrangement of Rvs on endocytic tubes. 

\subsection{The BAR and SH3 domain likely function together to localize Rvs}
Full-length Rvs is able to localize to cortical patches without the membrane-dependent interaction of the BAR domain (Fig. Full-length RVS in sla2del and LatA treatment shows cortical patches). This indicates that the SH3 domain is able to mediate recruitment of a cluster of Rvs molecules, and then disassemble this cluster.  The independent ability of the SH3 domain to localize and disassembly protein complexes is surprising, since SH3 domains are known so far to mediate protein-protein interaction and self-regulation of activation states. 

Unexpectedly, the lack of SH3 domain affects the recruitment of Rvs to the plasma membrane, even though the amount of protein expressed is the same (from analyzing cytoplasmic signal in Rvs167-GFP vs rvs167Δ SH3-GFP cells, not shown). This indicates that the SH3 domain increases the efficiency of recruitment of Rvs to invaginated tubes. Transient cortical patches in LatA treated cells expressing Rvs167-GFP that are absent in LatA cells expressing rvs167Δ SH3-GFP (Fig.1A) suggests that the former patches are caused by an interaction mediated by the SH3 domain

\subparagraph{What recruits the SH3 domain?}
SH3 interaction with the an endocytic binding partner could recruit Rvs to sites. Many such interaction partners have been proposed; Abp1 interaction with the Rvs167 SH3 domain has been shown27,28, Las1729,30, Cmd131, type I myosins32, Vrp127, which recruits the myosins, have all emerged as potential candidates, and are being studied as potential targets of the Rvs167 SH3 domain.


\subsection{Rvs does not form a tight scaffold on yeast membrane tubes}
Cryo EM structures of mammalian BAR proteins have suggested that the Rvs complex forms a similar helical scaffold with lateral interactions between adjacent BAR domains on invaginated membrane tubes. In the Rvs overexpression strains, Rvs can be recruited in much higher numbers, and at a much faster rate to the membrane than in WT cells, but appears to have similar disassembly dynamics as in the WT (Fig.5). The atypical, sharp decay fluorescent signal indicates that all of the Rvs on the membrane is suddenly released, consistent with the idea of a scaffold that breaks upon vesicle scission, releasing all the membrane-bound BAR protein. The decay in the 4x Rvs strain suggests that all the Rvs is also bound to the membrane, and since the membrane is now able to accommodate about 1.6x the amount of BAR protein as the WT on the same amount of membrane, it appears that in the WT case, a tight helix that covers the entire tube was not likely to be formed: adding molecules to such a tube would result in a change in at least disassembly dynamics.

\subsection{Rvs is recruited as dimers to endocytic sites}

\section{What causes membrane scission?}

\subsection{Dynamin does not not drive scission}
Some studies have suggested that Vps1 does localize to endoytic sites, and affects the scission mechanism: Nannapaneni et al.,23 find that the lifetime of Las17, Sla1, Abp1 increase in the absence of Vps1. Rooij et al.,24. find that Rvs167 lifetimes increase, and are recruited in fewer patches to the cell cortex. On the other hand, vps1Δ did not increase the scission failure rate of rvs167Δ in other studies25, and did not colocalize with other endocytic proteins26. We see a slight but insignificant increase in Rvs167 lifetimes in vps1Δ cells. If Vps1 was to affect scission, the number of failed scission events should increase in vps1Δ cells, and increase the lengths of invaginated tubes, but we do not find so. The lack of influence of Vps1 on coat and scission dynamics shows that membrane scission is not dependent on a dynamin interaction. Vp1 tagged with super-folded GFP and imaged in TIRF does not form cortical patches that co-localize with Abp1-mCherry (data from Andrea Picco, not shown). That tagging does not cause the slight growth defect of the vps1Δ cells is seen in dot spot assays. GFP-tagging could however, affect the recruitment of Vps1 to endocytic sites while maintaining its role in other cellular processes like vesicular trafficking. Outside of potential issues recruiting the protein to endocytic sites, membrane movement and scission dynamics are unchanged in the absence of Vps1, suggesting that even if it recruited to sites, it is not necessary for Rvs localization or function. 


\subsection{Lipid hydrolysis is not the primary cause of membrane scission}
Synaptojanin-mediated scission model19 predicts that first, vesicle scission would occur at the top of the invaginated tube, at the interphase of the hydrolyzed and non-hydrolyzed lipid. Kukulski et al.,16 have shown that vesicles undergo scission at 2/3 the invagination depths: that is, vesicles generated by lipid hydrolysis based line tension would be much smaller than have been seen. Second, removing forces generated by lipid hydrolysis should then increase the invagination lengths measured. Deletion of Inp51 and Inp52 do not change the invagination depths at which scission occurs, as measured from the maximum movement of the averaged Sla1 trajectory. Protein friction does not appear to contribute significantly to membrane scission, since invagination depths do not change if more BAR protein is localized to membrane tubes.

\subsection{ Actin polymerization generates enough force for membrane scission}

\section{What is the function of Rvs}
\subsection{Rvs scaffolds membrane pore}
Sla1 in rvs167Δ cells undergoes scission at short invagination lengths of about 60nm (Fig.4) , compared to the WT lengths of 140nm16; Rvs167 is required at membrane tubes to prevent premature scission. This is consistent with the SH3 domain mediating actin forces to the invagination neck, causing scission, as well as with Rvs167 stabilizing the membrane invagination via membrane interactions of the BAR domain33. Since WT invagination depths are reproduced by overexpression of the BAR domain alone, we propose that localization of Rvs to the membrane tube stabilizes the membrane pore, and allows deep invaginations to grow until actin polymerization produces enough forces to sever the membrane and cause scission. Here, the forces are generated entirely by actin polymerization, and the amount of force necessary is determined by the physical properties of the membrane.

\section{What is the role of other scission-stage proteins?}
\subsection{Inp52 is likely involved in uncoating vesicles after scission}
Deletion of Synaptojanin-like Inp52 does not affect the invagination depths of Sla1, but Sla1 patches persist for longer after scission in the inp52Δ than in WT cells, as do patches of Rvs167, indicated by the arrows in Fig.2. Both delays suggest that rather than the scission time-point, post- scission disassembly of proteins from the vesicle is inhibited by the deletion, and that Inp52 plays a role in recycling endocytic proteins to the plasma membrane.






