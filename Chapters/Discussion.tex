
\chapter{Discussion}    
\label{Ch:discussion}
The recruitment and function of the Rvs complex in the last scission stage of endocytosis has been explored in this work, as well as some previously untested proposals for how membrane scission could be effected in yeast endocytosis. 
I propse that Rvs localizes by interactions of the BAR domains of the Rvs complex with invaginated membranes, and that the SH3 domain is required for efficient and timely recruitment of Rvs to sites. Arrival of Rvs on membrane tubes then scaffolds the membrane tube and prevents membrane scission, in a manner that depends on recruitment of a critical number of Rvs molecules, till actin forces rupture the membrane, causing vesicle scission. 

Here I discuss the main findings of this thesis in support of these propositions.

\section{Recruitment of Rvs to endocytic sites}
Rvs is relatively short-lived protein at endocytic sites, recruited in the last stage, to membrane tubes: timing and position appear to be tightly regulated. FCS measurements have shown that the cytoplasmic content of Rvs167, as well as that of Rvs161 is quite high compared to other endocytic proteins: many early proteins, and several of the WASP/ Myosin module proteins like Las17, Vrp1, type1 myosins, are measured at 80-240nM, while cytoplasmic intensity of 161 is 721nM, and 167 is measured at 354nM. In spite of this, relatively few numbers of Rvs are recruited to endocytic sites, suggesting that cytoplasmic concentration alone does not determine the recruitment dynamics. Comparison between FCS measurements of cytoplasmic intensity for different endocytic proteins, and their recruitment to the cortex indicates low correlation between the two, perhaps unsurprisingly, requiring that other directed mechanisms recruit various proteins in a timed and efficient manner. In the case of Rvs, both timing and efficiency appear crutial to its function, the question is now what confers both.  

\subsection{Amount of Rvs}

\subsubsection{A limit for how much Rvs can be recruited to the membrane}
A change in disassembly dynamics is seen, however, in Rvs duplication in haploid cells. In this case, an even higher amount of Rvs is recruited to cells than in the WT: the maximum number of molecules recruited before scission is 178 +/- 7.5 compared to 113.505 +/- 5.2 yields a 1.57x recuitment of protein to membrane tubes. Here, the disassembly of Rvs after scission is delayed, and would suggest that the excess protein is not directly on the membrane. The excess Rvs either interacts with the actin network via the SH3 domain, or interacts with other Rvs dimers. Currently, I am not able to distinguish between the two, since the SH3 interaction partner is currently undetermined, and the arrangement of Rvs on the membrane is currently unknown. BAR-BAR interactions have been observed for other BAR proteins, albeit via lateral interactions at the tips of the curved structure, between apposed BAR domain. The concave face of BAR domains has been shown to interact with the membrane, and interactions that allow concentric arrangement of BAR domains are not seen before and are unlikely, but perhaps still possible.

Whatever the arrangement of the Rvs complex on the membrane, that the disassembly dynamics is changed in the case of 1.6x Rvs, compared to WT, and 1.4x Rvs suggests that there is a limit to how much Rvs can assemble on the tube without a change in protein-protien or protein-membrane interaction. Why there is a difference between recruitment of Rvs in the diploid and haploid case is uncertain. Diploid cells do not double in volume compared to haploids: in normal growth conditions, the volume of the diploid cell various between 1.57x that of the haploid cell, and the average cell surface area increases by 1.4x. In the gene duplication case, we have two copies of Rvs that in principle should expressed at twice the haploid level. Cytoplasmic quantification, however, shows that the increase is 1.4x in the duplicated diploid case compared to the WT diploid case, as does the recruitment to endocytic sites. There is then 1.4x the protein in nearly 1.6x the cellular volume, resulting in a dilution of the protein content per unit volume of the cell, which could then explain the decreased recruitment. 

It appears that the recruitment is then proportionate to gene copy number, and protein content in the cell, but that this is likely not the only factor that influences recruitment. 

\subsubsection{The SH3 domain makes Rvs localization efficient}
Cellular expression alone does not determine how much Rvs gets recruited. Unexpectedly, the lack of SH3 domain affects the recruitment of Rvs to the plasma membrane, even though the amount of protein expressed is the same (from analyzing cytoplasmic signal in Rvs167-GFP vs rvs167Δ SH3-GFP cells, see methods), the amount of protein recruited to endocytic sites is nearly halved. This indicates that the SH3 domain increases the efficiency of recruitment of Rvs to invaginated tubes. Transient cortical patches in LatA treated cells expressing Rvs167-GFP that are absent in LatA cells expressing rvs167Δ SH3-GFP (Fig.1A) suggests that the former patches are caused by an interaction mediated by the SH3 domain. These patches suggest that the SH3 domain is able to cluster protein to the cell cortex, although the mechanism by which this occurs is not known. 

Full-length Rvs is able to localize to cortical patches without the membrane-dependent interaction of the BAR domain (Fig. Full-length RVS in sla2del and LatA treatment shows cortical patches). This indicates that the SH3 domain is able to mediate recruitment of a cluster of Rvs molecules, and then disassemble this cluster.  The independent ability of the SH3 domain to localize and disassembly protein complexes is surprising, since SH3 domains are known so far to mediate protein-protein interaction and self-regulation of activation states. 

\subsection{Timing of Rvs}

\subsection{The BAR- membrane curvature interaction}
It has been shown before that Rvs localizes to already invaginated membrane tubes, suggesting that its role in the last stage is not to drive or significantly influence membrane curvature. From the expected curved structure of the BAR dimer, it has so far been assumed that it is recruited by its preference for some membrane shapes over others. In the absence of membrane curvature, in sla2del cells, the Rvs BAR domain does not localize to cortical patches. This demonstrate for the first time that the Rvs BAR domain does indeed require membrane curvature to localize to cortical patches. Work on other BAR domains have so far not discovered a specific interaction with a lipid subtype, and have suggested that hydrophobic interactions are mediate this interaction. It is not so far clear how the Rvs BAR domains interacts with the membrane, and mutations of lipid interacting surfaces will be necessary to determine the type of interaction with the underlying lipids. It is also unclear how Rvs is arranged on the membrane tube. Although solved structures of BAR domains show high structural similarity in spite of low sequence similarity, no structure for the Rvs complex exits. The fact that this is a hetero- rather than homodimer suggests that the structure does not necessarily resemble that of Amphiphysin or Endophilin homodimers, and a high-resolution structure will be necessary to clarify the interaction and arrangement of Rvs on endocytic tubes. 

\subsubsection{The SH3 domain affects timing of Rvs recruitment: scisssion and actin disassembly are likely uncoupled}


\subsubsection{What recruits the SH3 domain?}
SH3 interaction with the an endocytic binding partner could help recruit Rvs to sites. Many such interaction partners have been proposed; Abp1 interaction with the Rvs167 SH3 domain has been shown27,28, Las1729,30, Cmd131, type I myosins32, Vrp127, which recruits the myosins, have all emerged as potential candidates, and are being studied as potential targets of the Rvs167 SH3 domain. Since the SH3 localizes to sites in an actin independent manner, the interaction candidate is likely one that does not require actin to localize, leaving Vrp1, Las17, Myo3/5. Las17 and type 1 mysoins localize to the base of the invagination, and do not move into the cytoplasm a significant amount during invagination. If one of these was the SH3 interaction partner, SH3 domains are then recruited at the base of the invagination, and then pushed up with membrane as the tube grows longer. Centroid tracking however, suggests that Rvs is accumulated all over the membrane tube without bias towards the base of the invagination: if this was the case, the centroid would move upwards rather than remain non-motile. It is possible for the SH3 to drive early recruitment and localization, which is then "switched off" as Rvs is clustered by the SH3 domain, and targeted recruitment via an interaction partner is no longer necessary. 

\subsection{Arrangement of Rvs}

\subsubsection{Rvs does not form a tight scaffold on yeast membrane tubes}
Cryo EM structures of mammalian BAR proteins have suggested that the BAR dimers of Rvs might form a similar helical scaffold with lateral interactions between adjacent BAR domains on invaginated membrane tubes. In the Rvs overexpression strains in diploids, Rvs can be recruited in much higher numbers, and at a much faster rate to the membrane than in WT cells, but appears to have similar disassembly dynamics as in the WT (Fig.5). The atypical, sharp decay fluorescent signal indicates that all of the Rvs on the membrane is suddenly released, consistent with the idea of a scaffold that breaks upon vesicle scission, releasing all the membrane-bound BAR protein. The decay in the 4x Rvs strain suggests that all the Rvs is also bound to the membrane, and since the membrane is now able to accommodate about 1.6x the amount of BAR protein as the WT on the same amount of membrane, it appears that in the WT case, a tight helix that covers the entire tube was not likely to be formed: adding molecules to such a tube would result in a change in at least disassembly dynamics.

\subsection{Rvs161 and Rvs167 are recruited as dimers to endocytic sites}
FCCS and genetic studies have both proposed that the Rvs complex is recruited to endocytic sites as heterodimers, although the deletions of one do not exactly match the deletion of the other. Deletion of Rvs161, for example, confers a defect in cell fusion, that is not present in the rvs167deletion. FCCS measurements have indicated that Rvs167 and Rvs161 form stable heterodimers in the cytoplasm, although they appear to be expressed at different concentration. Cytoplasmic concentration of 161 is expressed at 721nM by FCS, while 167 is measured at 354nM, suggesting a nearly two-fold increase in expression of 161. Overexpression of 167 in the duplicated strain, however, does not lead to increased recruitment of that protein to endocytic sites, unless matched by overexpression of Rvs161. In the case of overexpression of BAR, without overexpression of 161, however, there is an increased recruitement of BAR to sites. Cytoplasmic background quantification has shown only a moderate decrease in the expression of BAR, compared to full-length Rvs167. This leads to a conundrum of recruitment: how does more BAR lead to more protein, while more 167 does not? I propose that 

\subsubsection{Contribution of the N-amphiphatic helix and GPA region}
The influence of the N-terminal amphiphatic helix (N-helix) and the unstructured GPA region have not been studied in this work. Preliminary results (not shown) indicate that removal of the N-helix does not prevent recruitment of endocytic sites. Other work has shown that N-helix is not necessary for localization of the protein, except in high salt conditions, and could aid clustering of the protein at endocytic sites in normal growth conditions. The GPA region 


\section{What causes membrane scission?}

\subsection{Dynamin does not not drive scission}
Some studies have suggested that Vps1 does localize to endoytic sites, and affects the scission mechanism: Nannapaneni et al.,23 find that the lifetime of Las17, Sla1, Abp1 increase in the absence of Vps1. Rooij et al.,24. find that Rvs167 lifetimes increase, and are recruited in fewer patches to the cell cortex. On the other hand, vps1Δ did not increase the scission failure rate of rvs167Δ in other studies25, and did not colocalize with other endocytic proteins26. We see a slight but insignificant increase in Rvs167 lifetimes in vps1Δ cells. If Vps1 was to affect scission, the number of failed scission events should increase in vps1Δ cells, and increase the lengths of invaginated tubes, but we do not find so. The lack of influence of Vps1 on coat and scission dynamics shows that membrane scission is not dependent on a dynamin interaction. Vp1 tagged with super-folded GFP and imaged in TIRF does not form cortical patches that co-localize with Abp1-mCherry (data from Andrea Picco, not shown). That tagging does not cause the slight growth defect of the vps1Δ cells is seen in dot spot assays. GFP-tagging could however, affect the recruitment of Vps1 to endocytic sites while maintaining its role in other cellular processes like vesicular trafficking. Outside of potential issues recruiting the protein to endocytic sites, membrane movement and scission dynamics are unchanged in the absence of Vps1, suggesting that even if it recruited to sites, it is not necessary for Rvs localization or function. 


\subsection{Lipid hydrolysis is not the primary cause of membrane scission}
Inp51 is not seen to localize, but cytoplasmic signal is low, suggesting low levels of expression. 

Synaptojanin-mediated scission model19 predicts that first, vesicle scission would occur at the top of the invaginated tube, at the interphase of the hydrolyzed and non-hydrolyzed lipid. Kukulski et al.,16 have shown that vesicles undergo scission at 2/3 the invagination depths: that is, vesicles generated by lipid hydrolysis based line tension would be smaller than have been seen. Second, removing forces generated by lipid hydrolysis by deleting synaptojanins should then increase the invagination lengths measured. Deletion of Inp51 and Inp52 do not change the invagination depths at which scission occurs, as measured from the maximum movement of Sla1. That the position of the vesicle formed is also unchanged is indicated by the magnitude of the jump into the cytoplasm of the Rvs complex. 

There are some changes, however in the synaptojanin deletion strains. FIrst, in the inp51del strain, Rvs assembly is slightly slower than that of the WT. Thus far, it is unclear what this means. Rvs centroid persists after scission for about a second longer than the WT does, indicating that disassembly of Rvs on the base of the newly formed vesicle is delayed. In the inp52del strain, about 12\% of Sla1-GFP tracks do not move into the cytoplasm and undergo scission. Although this is only half of the failed scission rate of the rvs167del cells, it could indicate that there is a moderate influence of inp52 on scission dynamics. In the inp5152del strain, most Rvs patches do not show the sharp jump into the cytoplasm. Membrane morphology is hugely aberrant in these cells, complicating the interpretation of this data. Electron microscopy shows long, undulating membrane invaginations, with multiple endocytic sites that are assembled and disassembled. Whether the Rvs complex localizes at the membrane invaginations could be clarified by CLEM or superresolution microscopy. Local lipid changes are so far not understood, as a native probe for lipids remains lacking. 



\subsection{Protein friction does not drive membrane scission}
Protein friction does not appear to contribute significantly to membrane scission, since invagination depths do not change if more BAR protein is localized to membrane tubes.

\subsection{ Actin polymerization generates enough force for membrane scission}

\section{What is the function of Rvs}
\subsection{Rvs scaffolds membrane pore}
Sla1 in rvs167Δ cells undergoes scission at short invagination lengths of about 60nm (Fig.4) , compared to the WT lengths of 140nm16; Rvs167 is required at membrane tubes to prevent premature scission. This is consistent with the SH3 domain mediating actin forces to the invagination neck, causing scission, as well as with Rvs167 stabilizing the membrane invagination via membrane interactions of the BAR domain33. Since WT invagination depths are reproduced by overexpression of the BAR domain alone, we propose that localization of Rvs to the membrane tube stabilizes the membrane pore, and allows deep invaginations to grow until actin polymerization produces enough forces to sever the membrane and cause scission. Here, the forces are generated entirely by actin polymerization, and the amount of force necessary is determined by the physical properties of the membrane.

\section{What is the role of other scission-stage proteins?}
\subsection{Inp52 is likely involved in uncoating vesicles after scission}
Deletion of Synaptojanin-like Inp52 does not affect the invagination depths of Sla1, but Sla1 patches persist for longer after scission in the inp52Δ than in WT cells, as do patches of Rvs167, indicated by the arrows in Fig.2. Both delays suggest that rather than the scission time-point, post- scission disassembly of proteins from the vesicle is inhibited by the deletion, and that Inp52 plays a role in recycling endocytic proteins to the plasma membrane.


\section{Model for membrane scisison}



