
\chapter{Discussion}    
\label{Ch:discussion}

The recruitment and function of the Rvs complex in the last scission stage of endocytosis has been explored in this work, as well as some previously untested proposals for how membrane scission could be effected in yeast endocytosis. 
I propose that Rvs localizes by interactions of the BAR domains of the Rvs complex with invaginated membranes, and that the SH3 domain is required for efficient and timely recruitment of Rvs to sites. Arrival of Rvs on membrane tubes then scaffolds the membrane tube and prevents membrane scission, in a manner that depends on recruitment of a critical number of Rvs molecules, till actin forces rupture the membrane, causing vesicle scission. 

Here I discuss the main findings of this thesis in support of these propositions.

\section{Recruitment of Rvs to endocytic sites}
Rvs is relatively short-lived protein at endocytic sites, recruited in the last stage, to membrane tubes1–3: timing and position appear to be tightly regulated. FCS measurements have shown that the cytoplasmic content of Rvs167, as well as that of Rvs161 is quite high compared to other endocytic proteins4: many early proteins, and several of the later WASP/ Myosin module like Las17, Vrp1, type1 myosins, are measured at 80-240nM, while cytoplasmic intensity of 161 is 721nM, and 167 is measured at 354nM. In spite of this, relatively few numbers of Rvs are recruited to endocytic sites, suggesting that cytoplasmic concentration alone does not determine recruitment dynamics. Comparison between FCS measurements of cytoplasmic concentration for different endocytic proteins, and their recruitment to the endocytic sites indicates low correlation between the two, perhaps unsurprisingly, requiring that other directed mechanisms recruit proteins in a timed and efficient manner. In the case of Rvs, both timing and efficiency appear crucial to its function, the question is what confers both.  

\subsection{Timing of localization and efficiency of recruitment}

\subsubsection{The BAR domain senses membrane curvature}
Rvs167 recruitment has been correlated against membrane shapes along the endocytic timeline. CLEM studies have shown that when Rvs arrival is plotted against the smallest angle between the two membrane sides2, which is 180 when the membrane is flat, and goes to 0 as the membrane becomes tubular, at the time of the recruitment of Rvs, this angle is 0. This indicates that Rvs is recruited only to tubular membrane shapes. From the curved structure of the BAR dimer, it has been assumed that it is recruited by its preference for some membrane shapes over others. In the absence of membrane curvature, in sla2del cells, the Rvs BAR domain does not localize to cortical patches. This demonstrates for the first time that this BAR domain does indeed sense and requires membrane curvature to localize to cortical patches. BAR recruitment by sensing membrane curvature would allow for this specificity. Work on BAR domains have so far not uncovered a specific interaction with a lipid subtype, and have suggested that hydrophobic interactions mediate this interaction. It is not clear how the Rvs BAR domains interacts with the membrane, and mutations of the lipid-binding surfaces are necessary to clarify the interaction with underlying lipids.

\subsubsection{BAR domain times recruitment of Rvs} 

Without the SH3 domain, that is in BAR cells, Rvs167 is able to localize to endocytic sites, and has a similar lifetime. In Fig3.4 A, B, Abp1 and Rvs167 in WT and BAR cells are aligned in time to the peak of the respective Abp1 fluorescent intensities. There are two striking differences when comparing the timing of proteins in the two cases: first, while WT Rvs arrives about 4 seconds after the arrival of Abp1, Rvs in BAR cells arrives only 6 seconds after Abp1 arrives. There is a time delay between Abp1 arrival and Rvs167 recruitment in BAR cells, confirmed by the TIRF measurement in 3.4D. 
The timing of Rvs arrival could be determined either by BAR interaction with membrane curvature, or by an SH3 interaction with an unknown binding partner. Rvs in BAR cells arrives when Sla1 has moved inwards 25-35nm, which is also the Sla1 distance moved when Rvs in WT arrives. This means that Rvs recruitment is delayed in BAR cells, since Sla1 moves inwards slower, and untill an invagination of a specific length is formed. This suggests that what times Rvs recruitment to endocytic sites is a particular membrane invagination length, and that this timing is provided by the BAR domain interaction. To be noted is that Sla1 is not directly at the plasma membrane, and the centroid of Sla1 sits about 20nm higher on the flat plasma membrane than Sla2. Therefore, a 25-35 distance of Sla1 would correspond to 45-55 nm of membrane invagination, by which point, the membrane is already parallel, and this is consistent with Rvs arrival at invaginated tubes. 

\subsubsection{SH3 domain regulates actin network} 
Another obersevation is that Abp1 accumulation is aberrant in BAR cells: it accumulates at the same rate as in WT cells till a point, and then slows dramatically, and stops at about half the WT Abp1 numbers.
The point the accumulation slows is interesting: it is at the time at which Rvs in WT cells arrives at endocytic sites. This could mean that SH3 domains are involved also in regulation of the actin network. Disassemby of Abp1 begins at the same time in BAR and WT cells, at shorter invagination lengths, and at lower Abp1 levels. This could indicate that Abp1 disassembly is triggered by an external "timer" (marked by the dashed red line in Fig.3.4B). 
	\vspace{5mm}
	
While in the WT cells, peaks of Abp1 and Rvs167 fluorescent intensity and consequent decay coincide, this is not true for the BAR cells. Rvs in these cells peaks several seconds after Abp1 intensity starts to drop. Abp1 drop indicates the disassembly of the actin network after vesicle scission. In WT cells, this drop coincides with a fast inward movement of the Abp1 centroid into the cytoplasm, and concomitant Rvs drop indicates membrane scission. A decoupling of the two could indicate that the actin disassembly and membrane scission do not coincide in BAR cells. The source of the Abp1 and Rvs167 peak mismatch in BAR cells is unclear. Since Rvs molecules in BAR cells sense membrane curvature, it is mostly likely that the BAR dimers are interacting with the membrane tube, so the decay in Rvs intensity in BAR cells most likely indicates membrane scission, also supporting the idea of an external timer for the disassembly of Abp1, and perhaps the entire actin network. This could imply that Abp1 is disassembled from the actin network before membrane scission. 

Abp1 is an activator of the Arp2/3 complex 

\subsubsection{The SH3 domain makes Rvs localization efficient}
Cellular expression alone does not determine how much Rvs gets recruited: as has been shown in Results 3.3, Rvs in BAR cells accumulates to about half its wild-type number, even though the same cytosolic concentration is measured (see methods). This indicates that the SH3 domain increases the efficiency of recruitment of Rvs to invaginated tubes. The decreased number of Rvs molecules recruited reduces the inward movement of Sla1 to similar to rvs167del cells, is discussed in the later section.

\subsubsection{The SH3 domain can assemble and disassemble Rvs molecules}
Transient cortical patches in LatA treated cells expressing Rvs167-GFP that are absent in LatA cells expressing rvs167Δ SH3-GFP (Fig.1A) suggests that the former patches are caused by an interaction mediated by the SH3 domain. These patches suggest that the SH3 domain is able to cluster protein to the cell cortex, although the mechanism by which this occurs is not known. 
	\vspace{5mm}
	
Full-length Rvs is able to localize to cortical patches without the membrane-dependent interaction of the BAR domain (Fig. Full-length RVS in sla2del and LatA treatment shows cortical patches). This indicates that the SH3 domain is able to mediate recruitment of a cluster of Rvs molecules, and then disassemble this cluster.  The independent ability of the SH3 domain to localize and disassembly protein complexes is surprising, since SH3 domains are known so far to mediate protein-protein interaction and self-regulation of activation states. 


\subparagraph{What does the SH3 domain interact with?}
SH3 interaction with the an endocytic binding partner could help recruit Rvs to sites. Many such interaction partners have been proposed; Abp1 interaction with the Rvs167 SH3 domain has been shown5,6, Las177,8, Cmd19, type I myosins10, Vrp15, which recruits the myosins, have all emerged as potential candidates, and are being studied as potential targets of the Rvs167 SH3 domain. Since the SH3 is able to localize to endocytic sites in an actin independent manner, the interaction candidate is likely one that does not require actin, leaving Vrp1, Las17, type1 myosins Myo3/5. Las17 and type 1 mysoins localize to the base of the invagination, and do not move into the cytoplasm a significant amount during invagination. If one of these was the SH3 interaction partner, SH3 domains are then recruited at the base of the invagination, and then pushed up with membrane as the tube grows longer. Centroid tracking however, suggests that Rvs is accumulated all over the membrane tube without bias towards the base of the invagination: if this was the case, the centroid would move upwards rather than remain non-motile. It is possible for the SH3 to drive early recruitment and localization, which is then "switched off" as Rvs is clustered by the SH3 domain, and targeted recruitment via an interaction partner is no longer necessary. 


\subsection{Arrangement of Rvs}

It is unclear how Rvs is arranged on the membrane tube. Although solved structures of BAR domains show high structural similarity in spite of low sequence similarity, no structure for the Rvs complex exits. The fact that this is a hetero- rather than homodimer suggests that the structure does not necessarily resemble that of Amphiphysin or Endophilin homodimers, and a high-resolution structure will be necessary to clarify the interaction and arrangement of Rvs on endocytic tubes

\subsubsection{Rvs does not form a tight scaffold on yeast membrane tubes}
Cryo EM structures of mammalian BAR proteins have suggested that the BAR dimers of Rvs might form a similar helical scaffold with lateral interactions between adjacent BAR domains on invaginated membrane tubes. In the Rvs overexpression strains in diploids, Rvs can be recruited in much higher numbers, and at a much faster rate to the membrane than in WT cells, but appears to have similar disassembly dynamics as in the WT (Fig.5). The atypical, sharp decay fluorescent signal indicates that all of the Rvs on the membrane is suddenly released, consistent with the idea of a scaffold that breaks upon vesicle scission, releasing all the membrane-bound BAR protein. The decay in the 4x Rvs strain suggests that all the Rvs is also bound to the membrane, and since the membrane is now able to accommodate about 1.6x the amount of BAR protein as the WT on the same amount of membrane, it appears that in the WT case, a tight helix that covers the entire tube was not likely to be formed: adding molecules to such a tube would result in a change in at least disassembly dynamics.

\subsubsection{A limit for how much Rvs can be recruited to the membrane}
A change in disassembly dynamics is seen, however, in Rvs duplication in haploid cells. In this case, an even higher amount of Rvs is recruited to cells than in the WT: the maximum number of molecules recruited before scission is 178 +/- 7.5 compared to 113.505 +/- 5.2 yields a 1.57x recuitment of protein to membrane tubes. Here, the disassembly of Rvs after scission is delayed, and would suggest that the excess protein is not directly on the membrane. The excess Rvs either interacts with the actin network via the SH3 domain, or interacts with other Rvs dimers. Currently, I am not able to distinguish between the two, since the SH3 interaction partner is currently undetermined, and the arrangement of Rvs on the membrane is currently unknown. BAR-BAR interactions have been observed for other BAR proteins, albeit via lateral interactions at the tips of the curved structure, between apposed BAR domain. The concave face of BAR domains has been shown to interact with the membrane, and interactions that allow concentric arrangement of BAR domains are not seen before and are unlikely, but perhaps still possible.

	\vspace{5mm}
Whatever the arrangement of the Rvs complex on the membrane, that the disassembly dynamics is changed in the case of 1.6x Rvs, compared to WT, and 1.4x Rvs suggests that there is a limit to how much Rvs can assemble on the tube without a change in protein-protien or protein-membrane interaction. Why there is a difference between recruitment of Rvs in the diploid and haploid case is uncertain. Diploid cells do not double in volume compared to haploids: in normal growth conditions, the volume of the diploid cell various between 1.57x that of the haploid cell, and the average cell surface area increases by 1.4x11. In the gene duplication case, we have two copies of Rvs that in principle should expressed at twice the haploid level. Cytoplasmic quantification, however, shows that the increase is 1.4x in the duplicated diploid case compared to the WT diploid case, as does the recruitment to endocytic sites. There is then 1.4x the protein in nearly 1.6x the cellular volume, resulting in a dilution of the protein content per unit volume of the cell, which could then explain the decreased recruitment. 

It appears that the recruitment is then proportionate to gene copy number, and protein content in the cell, but that this is not the only factor that influences recruitment. 


\subsection{Rvs161 and Rvs167 are recruited as dimers to endocytic sites}
FCCS and genetic studies have both proposed that the Rvs complex is recruited to endocytic sites as heterodimers, although the deletions of one do not exactly match the deletion of the other. Deletion of Rvs161, for example, confers a defect in cell fusion, that is not present in the rvs167deletion. FCCS measurements have indicated that Rvs167 and Rvs161 form stable heterodimers in the cytoplasm, although they appear to be expressed at different concentration. Cytoplasmic concentration of 161 is expressed at 721nM by FCS, while 167 is measured at 354nM, suggesting a nearly two-fold increase in expression of 161. Overexpression of 167 in the duplicated strain, however, does not lead to increased recruitment of that protein to endocytic sites, unless matched by overexpression of Rvs161. In the case of overexpression of BAR, without overexpression of 161, however, there is an increased recruitement of BAR to sites. Cytoplasmic background quantification has shown only a moderate decrease in the expression of BAR, compared to full-length Rvs167. This leads to a conundrum of recruitment: how does more BAR lead to more protein, while more 167 does not? I propose that 

\subsubsection{Contribution of the N-amphiphatic helix and GPA region}
The influence of the N-terminal amphiphatic helix (N-helix) and the unstructured GPA region have not been studied in this work. Preliminary results (not shown) indicate that removal of the N-helix does not prevent recruitment of endocytic sites. Other work has shown that N-helix is not necessary for localization of the protein, except in high salt conditions, and could aid clustering of the protein at endocytic sites in normal growth conditions. The GPA region is thought to function as the linker between the BAR and SH3 regions, and no other specific function is known. 

\section{What causes membrane scission?}


\subsection{Dynamin does not not drive scission}
Some studies have suggested that Vps1 does localize to endoytic sites, and affects the scission mechanism: Nannapaneni et al.,23 find that the lifetime of Las17, Sla1, Abp1 increase in the absence of Vps1. Rooij et al.,24. find that Rvs167 lifetimes increase, and are recruited in fewer patches to the cell cortex. On the other hand, vps1Δ did not increase the scission failure rate of rvs167Δ in other studies25, and did not colocalize with other endocytic proteins26. We see a slight but insignificant increase in Rvs167 lifetimes in vps1Δ cells. If Vps1 was to affect scission, the number of failed scission events should increase in vps1Δ cells, and increase the lengths of invaginated tubes, but we do not find so. The lack of influence of Vps1 on coat and scission dynamics shows that membrane scission is not dependent on a dynamin interaction. Vp1 tagged with super-folded GFP and imaged in TIRF does not form cortical patches that co-localize with Abp1-mCherry (data from Andrea Picco, not shown). That tagging does not cause the slight growth defect of the vps1Δ cells is seen in dot spot assays. GFP-tagging could however, affect the recruitment of Vps1 to endocytic sites while maintaining its role in other cellular processes like vesicular trafficking. Outside of potential issues recruiting the protein to endocytic sites, membrane movement and scission dynamics are unchanged in the absence of Vps1, suggesting that even if it recruited to sites, it is not necessary for Rvs localization or function. 


\subsection{Lipid hydrolysis is not the primary cause of membrane scission}
Inp51 is not seen to localize to the cellular cortex, but cytoplasmic concentration measured by FCS is low, suggesting low levels of expression that are likely undetected by our imaging protocol. Inp52 localizes to the top of invaginations right before scission, consistent with a role in vesicle formation. 

	\vspace{5mm}
However, the synaptojanin-mediated scission model19 predicts that first, vesicle scission would occur at the top of the invaginated tube, at the interphase of the hydrolyzed and non-hydrolyzed lipid. Kukulski et al.,16 have shown that vesicles undergo scission at 2/3 the invagination depths: that is, vesicles generated by lipid hydrolysis based line tension would be smaller than have been seen. Second, removing forces generated by lipid hydrolysis by deleting synaptojanins should then increase the invagination lengths measured. Deletion of Inp51 and Inp52 do not change the invagination depths at which scission occurs, as measured from the maximum movement of Sla1. That the position of the vesicle formed is also unchanged is indicated by the magnitude of the jump into the cytoplasm of the Rvs complex. 

	\vspace{5mm}
There are some changes, however in the synaptojanin deletion strains. FIrst, in the inp51del strain, Rvs assembly is slightly slower than that of the WT. Thus far, it is unclear what this means. Rvs centroid persists after scission for about a second longer than the WT does, indicating that disassembly of Rvs on the base of the newly formed vesicle is delayed. In the inp52del strain, about 12\% of Sla1-GFP tracks do not move into the cytoplasm and undergo scission. Although this is low compared to the failed scission rate of the rvs167del cells (close to 30\%), it could suggest a moderate influence of inp52 on scission dynamics. In the inp5152del strain, Rvs is accumulated at patches, but most Rvs patches do not show the sharp jump into the cytoplasm. Membrane morphology is hugely aberrant in these cells, complicating the interpretation of this data. 
Electron microscopy shows long, undulating membrane invaginations, with multiple endocytic sites that are assembled and disassembled. Where the Rvs complex localizes in these cells could be clarified by CLEM or superresolution microscopy. Large clusters of Rvs on the same invaginated tube would influence the molecule numbers acquired by this kind of analysis, and yield a higher number than at a single site (although what is actually happening at these patches is not actually clear). Rvs does, interestingly, assemble and disassemble. If there is no vesicle at these membrane, it would indicate that Rvs disassembly is decoupled from membrane scission.


\subsection{Protein friction does not drive membrane scission}
In Rvs duplicated strains, adding upto 1.6x the WT amount of Rvs to membrane tubes does not affect the length at which the membrane undergoes scission. The protein friction model introduced in Section 4.2.3 would suggest that if more BAR domains were added to the membrane tube, the frictional force generated as the membrane is pulled under it would increase, and the membrane would rupture "faster", that is, as soon as WT forces are generated on the tube. In the haploid duplication strain, the WT amount of Rvs is recruited at nearly -2 seconds, but scission does not occur at this time point. In the diploid strains, meanwhile, adding 1.4x the WT amount of Rvs does not change length of membrane scission. Decreasing the amount of Rvs from WT amounts, like in the 1x Rvs strain, however reduces the scission efficiency, and slightly reduces the inward movement of the membrane. 


\subsection{ Actin polymerization generates forces required for membrane scission}
Actin is essential for the formation of membrane invaginations in yeast cells (Ascough 2009). Cells treated with actin sequestring drugs like LatA fail to initiate endocytosis. In support of this, CLEM studies have shown that membrane invagination begins with the arrival of actin-binding Abp1 at endocytic sites, and only flat membranes are seen before its arrival. We find here, that irrespective of the number of BAR domains that are expressed or recruited to endocytic sites, membrane scission occurs at a specific amount of Abp1. This suggests that the actin network needs to grow to a specific amount, and generate a specific amount of force, at which point membrane scission occurs. 

\section{Function of the Rvs complex}

\subsection{Rvs scaffolds membrane pore}
The decreased coat movement in the rvs167Δ coat movement suggests that Rvs is required to prevent membrane scission. Rvs arrival then allows the membrane to continue to invaginate to WT lengths. Preventing membrane scission could be effected by either a BAR-interaction that would support the membrane tube and prevent collapse of the inner lipid membranes, or an SH3 dependent mechanism that would perhaps modulate actin dynamics, and prevent forces from being transmitted efficiently to the membrane neck, to prevent scission. From the BAR duplication experiments, longer invaginations than in the rvs167Δ or the BAR case can be reconstituted by simply adding more BAR domains. This supports the idea that preventing membrane scission by preventing fusion of the inner lipid bilayer in the membrane tube is the mechanism of action. 
BAR domains have been shown to induce membrane curvature in liposomes and prevent membrane scission. This requires a rigid membrane interacting surface and an inherently curved structure, both provided by the BAR domain. Although the BAR domain in this case does not induce curvature, it would stabilize the existing curvature and prevent membrane scission. 
\subparagraph{Role of the N-terminal helix}
The N-terminal amphiphatic helix has been shown to oppose the stabilizing effect of BAR domains by inducing membrane scission via shallow insertions of the helix into the membrane bilayer. The shallow insertions induce scission in a concentration dependent manner. In the case of yeast scission, since increasing the concentration of Rvs does not speed up the scission process, I do not expect that the N-helix plays a role in vesicle formation. The N-helix has also been shown to reqiured for membrane interaction, and could help the recruitment of Rvs to endocytic sites. The effect of the N-helix is currently being invesitaged.

\subsection{A critical amount of Rvs is required to allow invaginations to grow}

Sla1 in rvs167Δ cells undergoes scission at short invagination lengths of about 60nm (Fig.4), compared to the WT lengths of 140nm16; Rvs167 is required at membrane tubes to prevent premature scission. In BAR cells, half the WT amount of Rvs is recruited, and coat dynamics resemble that of rvs167Δ. This indicates that a threshold amount of Rvs is required to recapitulate WT endocytic dynamics. 

This is consistent with the SH3 domain mediating actin forces to the invagination neck, causing scission, as well as with Rvs167 stabilizing the membrane invagination via membrane interactions of the BAR domain33. Since WT invagination depths are reproduced by overexpression of the BAR domain alone, we propose that localization of Rvs to the membrane tube stabilizes the membrane pore, and allows deep invaginations to grow until actin polymerization produces enough forces to sever the membrane and cause scission. Here, the forces are generated entirely by actin polymerization, and the amount of force necessary is determined by the physical properties of the membrane.




\section{Role of other scission-stage proteins}
\subsection{Inp52 is likely involved in uncoating vesicles after scission}
Deletion of Synaptojanin-like Inp52 does not affect the invagination depths of Sla1, but Sla1 patches persist for longer after scission in the inp52Δ than in WT cells, as do patches of Rvs167, indicated by the arrows in Fig.2. Both delays suggest that rather than the scission time-point, post- scission disassembly of proteins from the vesicle is inhibited by the deletion, and that Inp52 plays a role in recycling endocytic proteins to the plasma membrane.


\section{Model for membrane scisison}


\section{Other potential scission mechanisms and open questions}

clustering induces scission 
cooperation between lipid hydrolysis and actin forces?


why these curvatures? specificity from the SH3 domain?
why does it come off
regulation of activity? phophorylation/ autoinhibition
