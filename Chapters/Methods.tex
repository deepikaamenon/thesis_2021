\chapter{Materials and Methods} % Main chapter title

\section{Materials} 
\subsection{Yeast strains used in this work}

\subsection{Media}
Media was kindly prepared by media kitchens of EMBL and University of Geneva BIochemistry Department, and by Anne-Sophie Riviera of the Kaksonen lab. Plates of all media were made by adding 2\% w/v bacto agar.
All media was autoclaved. Amino acid stocks were filter sterilised and added to the autoclaved media.


\subsection{Methods}
\subsubsection{Fluorescent tagging yeast with PCR casette insertion}
Tagging or deletion of endogenous genes was done by homologous integration of the product of a Polymerase Chain Reaction using appropriate primers and a plasmid containing a selection cassette and fluorescent tag, or only selection cassette for gene deletions. Primers were designed according to Janke et al, 2004. PCRs used the Velocity Polymerase for fluorescent tagging, and Q5 for gene deletions using the NAT casette. 
All fluorescently tagged genes have a C-terminus tag and are expressed endogenously.
Gene deletions and fluorescent tags are checked by PCR. Vps1del and gene duplications were confirmed by sequencing. 

\subsubsection{Live-cell imaging}
\subparagraph{Sample preparation for live imaging}
40 μL Concavalin A (ConA) was incubated on a coverslip for 10 minutes. 40 μL Yeast cells incubated overnight at 25C in imaging medium SC-TRP was added to the coverslip after removing the ConA, and incubated for another 10 minutes. Cells were then removed, adhered cells were washed 3x in SC-TRP, and 40 μL SC-TRP was finally added to the coverslip to prevent cells from drying. 

\subparagraph{Sample preparation for live imaging LatA and Sorbitol treated cells}
Cells went through the same procedure as above till the last washing step. Instead of SC-TRP, 100x diluted LatA, or Sorbitol to a concentration of 0.2M in SC-TRP was added to the adhered cells. For LatA experiments, cells were incubated in LatA for 10 minutes before imaging. For sorbitol treatments, cells were imaged within 5 minutes of adding sorbitol.

\subparagraph{Epifluorescent imaging for centroid tracking}
Live-cell imaging was performed as in Picco et al. All images were obtained at room temperature using an Olympus IX81 microscope equipped with a 100×/NA 1.45 PlanApo objective , with an additional 1.6x magnification lens and an EMCCD camera. The GFP channel was imaged using a 470/22 nm band-pass excitation filter and a 520/35 nm band-pass emission filter. mCherry epifluorescence imaging was carried out using a 556/20 nm band-pass excitation filter and a 624/40 band-pass emission filter. GFP was excited using a 488 nm solid state laser and mCherry was excited using a 561 nm solid state laser. Hardware was controlled using Metamorph software. For single-channel images, 80-120ms was used as exposure time. All dual-channel images were acquired using 250ms exposure time. Simultaneous dual-color images were obtained using a dichroic mirror, with TetraSpeck beads used to correct for chromatic abberation.

\subparagraph{Epifluorescent imaging for molecule number quantification}
Images were acquired as in Picco et al. Z-stacks of cells containing the GFP-tagged protein of interest, incubated along with cells containing Nuf2-GFP, were acquired using 400ms exposure using a mercury vapour lamp, on a CCD camera. 

\subparagraph{TIRF imaging}
TIRF microscopy was performed under similar conditions on an Olympus IX83 microscope. GFP was excited using a 488 nm solid state laser and mCherry was excited using a 561 nm solid state laser. Lasers, and shutters were controlled by Visitron Systems VS-Laser Control. VisiView software controlled the image acquisition and hardware-software feedback.
Images were processed using ImageJ, quantification was done on R.

\subsubsection{Image analysis }
Images were processed for background noise, particle detection, and tracking on ImageJ (http://imagej.nih.gov). Further analysis for centroid averaging, alignments between dual-color images and single channel images, for alignment to the reference Abp1 were done using scripts written in Matlab (Mathworks) and R (www.r-project.org), written originally by Andrea Picco, and modified by me. The methods for analysis can be found at Picco et al. 





\subparagraph{Sample preparation for CLEM}
			\mbox{}\\
\subsubitem
Freeze substitution
			\mbox{}\\
			stuff was stuck in a machine for a few days

\subparagraph{Electron Tomography}


