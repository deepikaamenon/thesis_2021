\chapter{Materials and Methods} % Main chapter title

\section{Materials} 
\subsection{Yeast strains used in this work}

\subsection{Media}
Media was kindly prepared by media kitchens of EMBL and University of Geneva BIochemistry Department, and by Anne-Sophie Riviera of the Kaksonen lab. Plates of all media were made by adding 2\% w/v bacto agar.
All media was autoclaved. Amino acid stocks were filter sterilised and added to the autoclaved media.


\subsection{Methods}
\subsubsection{Fluorescent tagging yeast with PCR casette insertion}
Tagging or deletion of endogenous genes was done by homologous integration of the product of a Polymerase Chain Reaction using appropriate primers and a plasmid containing a selection cassette and fluorescent tag, or only selection cassette for gene deletions. Primers were designed complimentary to a standardized 18bp region of the plasmid, and had 50bp overhang which was complementary to regions at beginning or end of the open reading frame of the gene of interest. PCRs used the Velocity Polymerase for fluorescent tagging, and Q5 for gene deletions using the NAT casette. 
All fluorescent tags in this work are at the C-terminus of the gene.

\subparagraph{Transformation of yeast strains with PCR casette}

Transformation with a PCR product was based on the lithium acetate method (Schiestl and Gietz, 1989).

Yeast cultures were set up in50 ml YPD at 30C on shakers for overnight incubation. Cultures were then diluted to optical density at 600 nm (OD600) of 0.2. The culture was then incubated at 30C on shaker till OD600 was measured between 0.8 and 1.0 (approximately three hours).

\subparagraph{Preparation of competent yeast cells} 
\mbox{}\\
Overnight yeast cultures at OD600 between 0.8 and 1 were centrifuged (500 g, 5 min, Room Temperature (RT)). The cell
pellet was washed in 10 mL H20. The pellet was washed in 5 ml SORB buffer and resuspended in 360 μL SORB, 40 μL salmon sperm DNA (ssDNA) (Sigma). ssDNA was denatured at 95C kept on ice. 50 μL aliquots were made while on ice and stored at -80C .

\subparagraph{Transformation of yeast cells with PCR product} 
\mbox{}\\
10 μL PCR product was added to thawed competent cells and incubated at RT for 30 min. 360 μL PEG buffer was added and the cells were heat-shocked in a water bath at 42C for 45 min. Cells that were to incorporate casettes that conferred antibiotic resistance were spun down (500 g, 2min, RT), resuspended in 5 ml YPD and incubated for 6 hours at 30C with shaking, spun down, resuspended in 250ml YPD and plated on selection plates. Cells with dropout cassettes were spun directly after heat-shock, resuspended in 200μL H2O, and plated directly on appropriate selection plates. Colonies that appeared on selection plates were streaked again to single colonies on selection plates, and then checked for correct integration of tag or deletion casette by PCR.

After around two days colonies were re-streaked to single cell colonies and correct integration was checked.
Yeast transformation of plasmids
1 μL of cells from a freshly grown YPD plate were resuspended in 0.1 M lithium acetate and 30 μL 1 M lithium acetate, 30 μL H20, 5 μL ssDNA, 200 ng plasmid and 240 μL 50\% w/v polyethylene glycol were added. The mixture was incubated at RT for 30 min. The cells were heat-shocked at 42 oC for 45 min. The cells were spun down (500 g, 2 min, RT), resuspended in 200 μL H20 and plated on the appropriate selection plate.

\subsubsection{Imaging and image analysis for centroid tracking and averaging}

\subparagraph{Sample preparation for live imaging}

\subparagraph{Sample preparation for live imaging Sorbitol treated cells}

\subparagraph{Epifluorescent and TIRF imaging}

\subparagraph{Sample preparation for CLEM}
\subsubitem

\subparagraph{Electron Tomography}


