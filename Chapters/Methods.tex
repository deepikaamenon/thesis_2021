\chapter{Materials and Methods} % Main chapter title

\section{Materials} 
\subsection{Yeast strains used in this work}

\subsection{Media}
Media was kindly prepared by media kitchens of EMBL and University of Geneva BIochemistry Department, and by Anne-Sophie Riviera of the Kaksonen lab. Plates of all media were made by adding 2\% w/v bacto agar.
All media was autoclaved. Amino acid stocks were filter sterilised and added to the autoclaved media.


\subsection{Methods}
\subsubsection{Fluorescent tagging yeast with PCR casette insertion}
Tagging or deletion of endogenous genes was done by homologous integration of the product of a Polymerase Chain Reaction using appropriate primers and a plasmid containing a selection cassette and fluorescent tag, or only selection cassette for gene deletions. Primers were designed complimentary to a standardized 18bp region of the plasmid, and had 50bp overhang which was complementary to regions at beginning or end of the open reading frame of the gene of interest. PCRs used the Velocity Polymerase for fluorescent tagging, and Q5 for gene deletions using the NAT casette. 
All fluorescent tags in this work are at the C-terminus of the gene.

\subparagraph{Transformation of yeast strains with PCR casette}

Transformation with a PCR product was based on the lithium acetate method (Schiestl and Gietz, 1989).

Yeast cultures were set up in50 ml YPD at 30C on shakers for overnight incubation. Cultures were then diluted to optical density at 600 nm (OD600) of 0.2. The culture was then incubated at 30C on shaker till OD600 was measured between 0.8 and 1.0 (approximately three hours).

\subparagraph{Preparation of competent yeast cells} 
\mbox{}\\
Overnight yeast cultures at OD600 between 0.8 and 1 were centrifuged (500 g, 5 min, Room Temperature (RT)). The cell
pellet was washed in 10 mL H20. The pellet was washed in 5 ml SORB buffer and resuspended in 360 μL SORB, 40 μL salmon sperm DNA (ssDNA) (Sigma). ssDNA was denatured at 95C kept on ice. 50 μL aliquots were made while on ice and stored at -80C .

\subparagraph{Transformation of yeast cells with PCR product} 
\mbox{}\\
10 μL PCR product was added to thawed competent cells and incubated at RT for 30 min. 360 μL PEG buffer was added and the cells were heat-shocked in a water bath at 42C for 45 min. Cells that were to incorporate casettes that conferred antibiotic resistance were spun down (500 g, 2min, RT), resuspended in 5 ml YPD and incubated for 6 hours at 30C with shaking, spun down, resuspended in 250ml YPD and plated on selection plates. Cells with dropout cassettes were spun directly after heat-shock, resuspended in 200μL H2O, and plated directly on appropriate selection plates. Colonies that appeared on selection plates were streaked again to single colonies on selection plates, and then checked for correct integration of tag or deletion casette by PCR.

After around two days colonies were re-streaked to single cell colonies and correct integration was checked.
Yeast transformation of plasmids
1 μL of cells from a freshly grown YPD plate were resuspended in 0.1 M lithium acetate and 30 μL 1 M lithium acetate, 30 μL H20, 5 μL ssDNA, 200 ng plasmid and 240 μL 50\% w/v polyethylene glycol were added. The mixture was incubated at RT for 30 min. The cells were heat-shocked at 42 oC for 45 min. The cells were spun down (500 g, 2 min, RT), resuspended in 200 μL H20 and plated on the appropriate selection plate.

\subsubsection{Imaging and image analysis }
Images were processed for background noise, particle detection, and tracking on ImageJ (http://imagej.nih.gov). Further analysis for centroid averaging, alignments between dual-color images and single channel images, for alignment to the reference Abp1 were done using scripts written in Matlab (Mathworks) and R (www.r-project.org), written originally by Andrea Picco, and modified by me. 

\subparagraph{Sample preparation for live imaging}
40 μL Concavalin A (ConA) was incubated on a coverslip for 10 minutes. 40 μL Yeast cells incubated overnight at 25C in imaging medium SC-TRP was added to the coverslip after removing the ConA, and incubated for another 10 minutes. Cells were then removed, adhered cells were washed 3x in SC-TRP, and 40 μL SC-TRP was finally added to the coverslip to prevent cells from drying. 

\subparagraph{Sample preparation for live imaging LatA and Sorbitol treated cells}
Cells went through the same procedure as above till the last washing step. Instead of SC-TRP, 100x diluted LatA, or Sorbitol to a concentration of 0.2M in SC-TRP was added to the adhered cells. For LatA experiments, cells were incubated in LatA for 10 minutes before imaging. For sorbitol treatments, cells were imaged within 5 minutes of adding sorbitol.


\subparagraph{Epifluorescent imaging for centroid tracking}
Live-cell imaging was performed at room temperature using an Olympus IX81 microscope equipped with a 100×/NA 1.45 PlanApo objective , with an additional 1.6x magnification lens. IThe GFP channel was imaged using a 470/22 nm band-pass excitation filter and a 520/35 nm band-pass emission filter. mCherry epifluorescence imaging was carried out using a 556/20 nm band-pass excitation filter and a 624/40 band-pass emission filter. GFP was excited using a 488 nm solid state laser and mCherry was excited using a 561 nm solid state laser. The Orca-ER camera, and electronic shutters and filter wheels were controlled by Metamorph software.

\subparagraph{Epifluorescent imaging for centroid tracking}
TIRF microscopy was performed undersimilar conditions on an Olympus IX83 microscope. GFP was excited using a 488 nm solid state laser and mCherry was excited using a 561 nm solid state laser. Lasers, and shutters were controlled by Visitron Systems VS-Laser Control. Imaging was controlled  by VisiView software.

\subparagraph{Sample preparation for CLEM}
			\mbox{}\\
\subsubitem
Freeze substitution
			\mbox{}\\
			stuff was stuck in a machine for a few days

\subparagraph{Electron Tomography}


