\chapter{Materials and Methods} % Main chapter title

\section{Materials} 
\subsection{Yeast strains used in this work}

\subsection{Media}
Media was kindly prepared by media kitchens of EMBL and University of Geneva BIochemistry Department, and by Anne-Sophie Riviera of the Kaksonen lab. Plates of all media were made by adding 2\% w/v bacto agar.
All media was autoclaved. Amino acid stocks were filter sterilised and added to the autoclaved media.



\subsection{Methods}
\subsubsection{Fluorescent tagging yeast with PCR casette insertion}
Tagging or deletion of endogenous genes was done by homologous integration of the product of a Polymerase Chain Reaction using appropriate primers and a plasmid containing a selection cassette and fluorescent tag, or only selection cassette for gene deletions. Primers were designed complimentary to a standardized 18bp region of the plasmid, and had 50bp overhang which was complementary to regions at beginning or end of the open reading frame of the gene of interest. PCRs used the Velocity Polymerase for fluorescent tagging, and Q5 for gene deletions using the NAT casette. 
All fluorescent tags in this work are at the C-terminus of the gene.

\subsubsection{Transformation of yeast strains with PCR casette}

Transformation with a PCR product was based on the lithium acetate method (Schiestl and Gietz, 1989).

Yeast cultures were set up in50 ml YPD at 30C on shakers for overnight incubation. Cultures were then diluted to optical density at 600 nm (OD600) of 0.2. The culture was then incubated at 30C on shaker till OD600 was measured between 0.6 and 1.0 (approximately three hours).

\subparagraph{Preparation of competent yeast cells} 

The yeast culture was spun down (500 g, 5 min, Room Temperature (RT)). The cell
pellet was washed in 10 mL H20 and spun down. The cell pellet was washed in 5 ml 115
6. Materials and Methods
6. Materials and Methods
SORB buffer (section 6.1.4) and spun down. Finally the cell pellet was resuspended in 360 μL SORB, 40 μL salmon sperm DNA (ssDNA) (Sigma), which had previously been denatured for 10 min at 95 oC then kept on ice. 50 μL aliquots were made and stored at 80 oC.

• Transformation
The PCR product was purified using the isolate II PCR and Gel kit (Bioline). 10 μL of purified PCR product was added to thawed competent cells and then incubated at RT for 10 min. 360 μL PEG buffer (section 6.1.4) was added and the cells were heat-shocked in a water bath at 42 oC for 45 min. The cells were then spun down (500 g, 2min, RT). The cells were resuspended in 5 ml YPD and incubated for several hours at 30 oC with shaking for recovery. They were then plated on the appropriate selection plate. After around two days colonies were re-streaked to single cell colonies and correct integration was checked.
Yeast transformation of plasmids
1 μL of cells from a freshly grown YPD plate were resuspended in 0.1 M lithium acetate and 30 μL 1 M lithium acetate, 30 μL H20, 5 μL ssDNA, 200 ng plasmid and 240 μL 50\% w/v polyethylene glycol were added. The mixture was incubated at RT for 30 min. The cells were heat-shocked at 42 oC for 45 min. The cells were spun down (500 g, 2 min, RT), resuspended in 200 μL H20 and plated on the appropriate selection plate.
