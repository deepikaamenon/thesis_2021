\chapter{Introduction} % Main chapter title

\label{Ch:Aims} % Change X to a consecutive number; for referencing this chapter elsewhere, use \ref{ChapterX}
some other shit happened

More than 50 different proteins are involved in clathrin-mediated endocytosis. At endocytic sites, they assemble into a small, complex and dynamic macromolecular machinery. Although individual components have been identified and well-characterized during decades of research, their structural organization is poorly understood. In my PhD project, I proposed that single-molecule localization based superresolution microscopy provides both the molecular specificity and necessary spatial resolution to study how proteins are arranged \textit{in situ} within the endocytic machinery. More specifically, I addressed the following questions:

\begin{itemize}
	\item Is localization microscopy suitable to study endocytosis in budding yeast?
\end{itemize}

The recently developed technique of localization microscopy critically depends on dense and specific fluorescent labeling of the cellular structure of interest with a dye suitable for localization microscopy. In the first part of my project, I established an optimized sample preparation pipeline to enable high quality dual-color localization microscopy of yeast cells. These efforts are described in section \ref{LocMic_yeast}.

\begin{itemize}
	\item How are proteins arranged within the endocytic machinery?
\end{itemize}

Using the methods I established, I studied the structural organization of endocytic proteins. First, I focused on their radial distribution, which had not been possible to study with live-cell fluorescence microscopy or EM. To increase the statistical power of my approach, I set up a high-throughput imaging and quantitative data analysis pipeline together with Joran Deschamps, another PhD student in the lab, which allowed me to image the organization of proteins at thousands of endocytosis sites. I found that endocytic proteins have an intricate radial organization, which mirrors their functional context and modularity. I also discovered that a set of proteins pre-pattern where actin polymerization will start, which likely supports the mechanistic robustness of the mobile phase. Results of these experiments are described in section \ref{CME_radial}.

\begin{itemize}
	\item How is the endocytic machinery organized during the highly dynamic mobile phase of endocytosis?
\end{itemize}

Visualizing the endocytic machinery during the most dynamic mobile phase with high spatial resolution is methodologically challenging, and has not been possible so far. Here, I proposed that localization microscopy, despite being a comparably slow technique, can overcome this gap. For this, I developed an approach to infer the time point of many fixed, unsynchronized endocytic sites by quantitatively analyzing their structures, and integrating the superresolution images with centroid trajectories determined in living cells \citep{Picco:2015iv}, and time-resolved membrane shapes measured by CLEM \citep{Kukulski:2012jl}. This has allowed me to visualize the highly dynamic formation of the actin network, which provides the force to invaginate the membrane, in relation to the zone where new filaments are nucleated. Results of this work are presented in section \ref{Res_Timeresolved}.