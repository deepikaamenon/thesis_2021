\chapter{Introduction} % Main chapter title

\label{Ch:Aims} % Change X to a consecutive number; for referencing this chapter elsewhere, use \ref{ChapterX}

\section{Endocytosis and cellular signalling}
The plasma membrane serves as the defining barrier between the internal and external cell, thus creating cellular identity, facilitates evolution out of the primordial soup into a defined structure that can regulate entry of signals into the cell. In eukaryotes, and with increasing complexity, in multicellular eukaryotes, tuning cellular response to external signals has resulted in a complex network of signaling pathways, and a tight coupling of these pathways with the process of endocytosis. Endocytosis is defined as the uptake of molecules too big to pass through the plasma membrane. It involves the invagination of the plasma membrane into a cargo-filled tube, and culminates in the severing of this tube to form a cargo-filled vesicle. Apart from internalizing cargo, it allows regulation of the plasma membrane itself: its lipid and protein composition, and therefore many physical and biochemical properties like tension, rigidity, surface receptor composition and localization. Cargo taken up by endocytic pathways include these surface receptors, that are transported elsewhere in the cell for degradation or as part of a signaling cascade, forming the link between cell signaling and endocytosis.  
Somewhat dramatically, endocytosis “constitutes the major communications infrastructure of the cell. As such, it governs almost all aspects of the relationships of the cell with the extracellular environment and of intracellular communication. Its evolution constitutes, arguably, the major driving force in the evolution of prokaryotic to eukaryotic organisms”1. 
Other components of the secretary pathway like the Golgi apparatus and endoplasmic reticulum undergo similar transitions of the bounding membrane to transmit signals and other cargo across the cell, and have mechanistic similarities. \par 

\vspace{5mm}
Although many early discoveries relating to endocytic pathways were first identified in mammalian cell types2,3, description of endocytosis in S.cerevisiae4 marked the beginning of important discoveries being made in the yeast system that would then be verified in mammalian cells. The ease of genetic manipulation, establishment of the yeast genome, and relative simplicity of endocytic pathways- there is only one -
 drove several discoveries using yeast as a model system that were later verified in mammalian cells.



\section{Clathrin-mediated endocytosis}
	Clathrin-mediated endocytosis (CME).
Many different endocytic pathways that facilitate the internalization of cargo at the plasma membrane exist, as depicted in Fig.1. Of them, Clathrin-mediated endocytosis (CME), is universal among eukaryotes and contributes to 90\% of cargo trafficked into the cell2. First identified by studying yolk uptake in mosquitos, ultrastructural studies of their oocytes (where the concentration of uptake events is high enough to be easily studied) identified a bristly coat formation on the cell membrane and similarly bristly vesicles, that then lost this coat and fused to eventually form yolk bodies in the mature oocyte3. The bristle is seen on several cell types, and was later identified as a lattice of a single highly conserved protein4. This protein was named Clathrin, derived from the latin word for lattice. Clathrin is formed of light and heavy chains incorporated into a triskelion that assemble into closed hexagonal and pentagonal structures on the inner leaflet of the plasma membrane. Clathrin-mediated endocytosis has, since four decades ago, been recognized has an ubiquitous mechanism of plasma membrane uptake in cell types ranging from the frog presynaptic membrane5 to rat vas deferens6. 

	
\section{Clathrin mediated endocytosis in Mammalian vs Yeast}
		\subsection{Clathrin is required for mammalian CME}
That the Clathrin lattice is responsible for remodeling the plasma membrane and selecting cargo was speculated in the first papers that noted the “bristly” coat6,11.  In multicellular organisms like C.elegans, clathrin depleted by RNAi result in decreased endocytic uptake in oocytes and dead progeny12, in D.melanogaster, deletion of Clathrin heavy chain results in embryonic lethality13. In Hela cells, knock down of the heavy chain by RNAi results in decrease in endocytosis by 80\%14; essentially, endocytosis fails in the absence of clathrin. The role of clathrin in the progression has been heavily debated, but its involvement itself has not. Although several genes involved in Clathrin-mediated endocytosis in yeast were found to be homologues of the mammalian machinery, early work in yeast revealed that clathrin was not necessary for endocytosis15. It became apparent that though the mammalian and yeast systems were mechanistically similar and most of the yeast endocytic proteins had mammalian homologues16, there are some significant differences. 

		\subsection{Actin forces are required for yeast CME}
Cortical actin patches were first seen in S.cerevisae, that were later established as endocytic sites from the colocolization of other endocytic proteins. While the mammalian CME uptake is heavily dependent on clathrin, the yeast system relies on actin and its proper organization for endocytosis17. Not only is actin itself necessary for the intiation of plasma membrane deformation18, coupling the endocytic coat to actin are necessary for internalization19,20.  The cell wall surrounding the plasma membrane in yeast cells has meant that the yeast cells are under high turgor pressure21, which would explain the high force requirement for membrane deformation in yeast. 

	
		\subsection{Clathrin mediated endocytosis in yeast is modular}
	CME in yeast involves the recruitment, interaction, and disassembly of over fifty proteins. This process involves a flexible initiation phase that establishes endocytic sites and selects cargo15, then a very stereotypic sequence of events nucleates actin, organizes the actin network, invaginates a membrane tube, and finally severs the membrane to produce cargo-filled vesicles16. Work on yeast has shown that the initiation of endocytic sites is independent of the recruitment of any one protein15, and revealed how the the components17 and organization of the actin machinery allows the formation of a membrane invagination18–20. However, the mechanism of the final stage of membrane scission that leads to vesicle formation remains unclear.

		\subparagraph{The early phase}
		\subparagraph{The actin phase}
		\subparagraph{The scission phase}
		
		
	\subsection{Membrane scission in mammalian cells}
		\subparagraph{Scission is dependent on dynamin} 
\mbox{} \\
		 In mammalian cells, membrane scission in endocytosis is effected by the combined action of BAR domain proteins like Endophilin and Amphiphysin and the GTPase dynamin 28.  Dynamin was discovered as a microtubule interacting protein 29, and since has been shown to have a pivotal role in membrane scission and fission at many different organelles across the cell. The importance of Dynamin was demonstrated in a temperature sensitive mutant of the Drosophila shibire gene, which results in paralysis of flies at the non-permissive temperature, from failure to form synaptic vesicles 30–32. Shibire codes multiple isoforms of dynamin that are differentially expressed across the organism 33. Deletion of dynamin isoforms results in initiation of clathrin-coated pits, but vesicle formation is disrupted, resulting in accumulation of a large number of long tubes 28. 

		 
		\subparagraph{Dynamin is an oligomeric GTPase}
\mbox{} \\ 
Dynamins consist of a GTPase domain, a stalk region, a bundle signalling element that acts as the linker between the GTPase domain and stalk, a PIP2 binding pleckstrin homology domain (PH) domain and a proline rich domain (PRD) that extends beyond the GTPase domain34. In-vitro, dynamin oligomerizes into helical structures with the PH domain apposed against the membrane, and the GTPase domain facing away 35,36. Dynamin within the helical structure undergoes a conformation change upon GTP hydrolysis that constricts the helix as well as the membrane tube under it, collapsing the inner leaflet of the bilayer membrane into a hemi-fusion state, finally resulting in membrane fission37.

	\subparagraph{Dynamin interacts with BAR proteins}
\mbox{} \\
Dynamin arrives at clathrin-coated pits via interaction with BAR proteins Endophilin and Amphiphysin28. BAR domain proteins are intrinsically curved protein dimers that have been named for the conserved module contained in their founding members, metazoan Amphiphysin/BIN and yeast proteins Rvs167, Rvs161. The BAR domain superfamily contains highly conserved structure across eukaryotes, in spite of little sequence similarity, and can be grouped into the classical BAR family, Fer–Cip4-homology-BAR (FBAR), and Inverse-BAR (IBAR) based on the curvature of the dimers. High resolution structural data has shown that BAR proteins can self-assemble into helical scaffolds on membrane tubes 38–40. BAR domains may thus induce curvature, stabilize curvature, or have a preference for specific membrane curvature that matches their intrinsic curvature. In addition to the BAR domain, most BAR proteins have additional motifs that mediate their interaction with membranes or other proteins: some BAR proteins like Endophilin, Amphiphysin have an N-terminal amphiphatic helix that is inserted into a membrane bilayer, phosophoinositide binding motifs like phox or pleckstrin homology (PH) domains direct BAR proteins to specific lipid enriched membranes, Endophilin/ Amphiphysin have Src homology 3 (SH3) domains that mediate protein-protein interaction. These SH3 regions act as a scaffold for the proline-rich domains of dynamin41. 

\vspace{5mm}
Dynamin and BAR proteins Endophilin and Amphiphysin interact via the dynamin PRD and BAR SH3 domains44–47. Endophilin recruitment is reduced in the absence of dynamin, and appears to inhibit the GTPase action of dynamin45,47,48, while dynamin recruitment is decreased without endophilin. Amphiphysin levels are unchanged in absence of dynamin, while deletion of amphiphysin results in increased recruitment and prolonged lifetimes of dynamin and absence of membrane scission45. These results suggest a role for amphiphysin for disassembly of dynamin involving GTP hydrolysis, and a role for endophilin in dynamin assembly, although the mechanistic interplay between the two BAR proteins with dynamin, and the sequence of events is not clear48,49. Although dynamin does not require BAR proteins to localize to clathrin-coated pits, both GTP hydrolysis and an interaction between the BAR proteins and dynamin is necessary for efficient vesicle scission45,50.





	\subsection{Membrane scission in yeast}
In yeast, none of the three dynamin-like proteins have proline-rich domains that could interact with the SH3 domain of BAR proteins, suggesting that scission is not driven by the mammalian mechanism. Here, the Amphiphysin/ Endophilin homologue is the heterodimeric complex Rvs161/16726 (Rvs), of which Rvs167 has an SH3 domain. Rvs arrives at endocytic sites in the late stage of the process, and disassembles rapidly at the time of membrane scission27. Deletion of Rvs results in failure of membrane scission in nearly 30\% of endocytic events16. This unique profile suggests that although Rvs is not necessary for scission, localization of the complex makes scission more efficient. The rapid disassembly of Rvs27, in the context of mammalian BAR protein data, has suggested that Rvs may form a similar scaffold on membrane tubes in yeast, whose disassembly is coupled to the scission step. What may regulate timing of scission in the absence of dynamin-BAR protein interaction however, has not been determined. Yeast cells are under high turgor pressure that makes forces from actin polymerization necessary for invagination28. There is therefore likely to be some interplay between scission-stage proteins and the actin network that could modulate the final shape transitions. 
	
		
	\section{BAR domain proteins}
		\subsection{Mammalian vs yeast BAR proteins}
		\subsection{The Rvs complex}		
		
		
	
%\begin{itemize}
%	\item Is localization microscopy suitable to study endocytosis in budding yeast?
%\end{itemize}
