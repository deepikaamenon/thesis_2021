\chapter{Introduction} % Main chapter title

\label{Ch:Aims} % Change X to a consecutive number; for referencing this chapter elsewhere, use \ref{ChapterX}


\section{Endocytosis and cellular signalling}
Endocytosis:
The plasma membrane serves as the defining barrier between the internal and external cell, thus creating cellular identity, facilitates evolution out of the primordial soup into a defined structure that can regulate entry of signals into the cell. In eukaryotes, and with increasing complexity, in multicellular eukaryotes, tuning cellular response to external signals has resulted in a complex network of signaling pathways, and a tight coupling of these pathways with the process of endocytosis. Endocytosis is defined as the uptake of molecules too big to pass through the plasma membrane. It involves the invagination of the plasma membrane into a cargo-filled tube, and results in membrane scission that forms a cargo-filled vesicle. Apart from internalizing cargo, it allows regulation of the plasma membrane itself: its lipid and protein composition, and therefore many physical and biochemical properties like tension, rigidity, surface receptor composition. Somewhat dramatically, endocytosis “constitutes the major communications infrastructure of the cell. As such, it governs almost all aspects of the relationships of the cell with the extracellular environment and of intracellular communication. Its evolution constitutes, arguably, the major driving force in the evolution of prokaryotic to eukaryotic organisms”1.



	\section{Clathrin mediated endocytosis}
	Clathrin-mediated endocytosis (CME).
Many different endocytic pathways that facilitate the internalization of cargo at the plasma membrane exist, as depicted in Fig.1. Of them, Clathrin-mediated endocytosis (CME), is universal among eukaryotes and contributes to 90 \% of cargo trafficked into the cell2. First identified by studying yolk uptake in mosquitos, ultrastructural studies of their oocytes (where the concentration of uptake events is high enough to be easily studied) identified a bristly coat formation on the cell membrane and similarly bristly vesicles, that then lost this coat and fused to eventually form yolk bodies in the mature oocyte3. The bristle is seen on several cell types, and was later identified as a lattice of a single highly conserved protein4. This protein was named Clathrin, derived from the latin word for lattice. Clathrin is formed of light and heavy chains incorporated into a triskelion that assemble into closed hexagonal and pentagonal structures on the inner leaflet of the plasma membrane. Clathrin-mediated endocytosis has, since four decades ago, been recognized has an ubiquitous mechanism of plasma membrane uptake in cell types ranging from the frog presynaptic membrane5 to rat vas deferens6. 

	
	\section{Mammalian vs Yeast endocytosis}
	
	
	\section{Clathrin mediated endocytosis is modular}
		\subsection{The early phase}
		\subsection{The actin phase}
		\subsection{The scission phase}
		
		
	\section{Membrane scission in mammalian cells}
	\section{Membrane scission in yeast}
	
		
	\section{BAR domain proteins}
		\subsection{Mammalian vs yeast BAR proteins}
		\subsection{The Rvs complex}		
		
		
	
%\begin{itemize}
%	\item Is localization microscopy suitable to study endocytosis in budding yeast?
%\end{itemize}
