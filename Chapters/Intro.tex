\chapter{Introduction} % Main chapter title

\label{Ch:Aims} % Change X to a consecutive number; for referencing this chapter elsewhere, use \ref{ChapterX}

Endocytosis is an ancient pathway that has likely contributed to the evolution of cellular diversity and specificity, allowing organization of signaling pathways, and plasma membrane homeostasis and composition. Of the many endocytic pathways, Clathrin-mediated endocytosis (CME) is universal among eukaryotes, and the only one in S.cerevisiae. CME in yeast involves the recruitment, interaction, and disassembly of over fifty proteins. This process involves a flexible initiation phase that establishes endocytic sites and selects cargo1, then a very stereotypic sequence of events nucleates actin, organizes the actin network, invaginates a membrane tube, and finally severs the membrane to produce cargo-filled vesicles2. Work on yeast has shown that the initiation of endocytic sites is independent of the recruitment of any one protein1, and revealed how the the components3 and organization of the actin machinery allows the formation of a membrane invagination4–6. However, the mechanism of the final stage of membrane scission that leads to vesicle formation remains unclear.


%	\section{Clathrin mediated endocytosis}
	
	\section{Mammalian vs Yeast endocytosis}
	
	
	\section{Clathrin mediated endocytosis is modular}
		\subsection{The early phase}
		\subsection{The actin phase}
		\subsection{The scission phase}
		
		
	\section{Membrane scission in mammalian cells}
	\section{Membrane scission in yeast}
	
		
	\section{BAR domain proteins }
		\subsection{Mammalian vs yeast BAR proteins}
		\subsection{The Rvs complex}		
		
		
	
%\begin{itemize}
%	\item Is localization microscopy suitable to study endocytosis in budding yeast?
%\end{itemize}
