\chapter{Introduction} % Main chapter title

\label{Ch:Aims} % Change X to a consecutive number; for referencing this chapter elsewhere, use \ref{ChapterX}


\section{Endocytosis and cellular signalling}
Endocytosis:
The plasma membrane serves as the defining barrier between the internal and external cell, thus creating cellular identity, facilitates evolution out of the primordial soup into a defined structure that can regulate entry of signals into the cell. In eukaryotes, and with increasing complexity, in multicellular eukaryotes, tuning cellular response to external signals has resulted in a complex network of signaling pathways, and a tight coupling of these pathways with the process of endocytosis. Endocytosis is defined as the uptake of molecules too big to pass through the plasma membrane. It involves the invagination of the plasma membrane into a cargo-filled tube, and results in membrane scission that forms a cargo-filled vesicle. Apart from internalizing cargo, it allows regulation of the plasma membrane itself: its lipid and protein composition, and therefore many physical and biochemical properties like tension, rigidity, surface receptor composition. Somewhat dramatically, endocytosis “constitutes the major communications infrastructure of the cell. As such, it governs almost all aspects of the relationships of the cell with the extracellular environment and of intracellular communication. Its evolution constitutes, arguably, the major driving force in the evolution of prokaryotic to eukaryotic organisms”1.



\section{Clathrin-mediated endocytosis}
	Clathrin-mediated endocytosis (CME).
Many different endocytic pathways that facilitate the internalization of cargo at the plasma membrane exist, as depicted in Fig.1. Of them, Clathrin-mediated endocytosis (CME), is universal among eukaryotes and contributes to 90\% of cargo trafficked into the cell2. First identified by studying yolk uptake in mosquitos, ultrastructural studies of their oocytes (where the concentration of uptake events is high enough to be easily studied) identified a bristly coat formation on the cell membrane and similarly bristly vesicles, that then lost this coat and fused to eventually form yolk bodies in the mature oocyte3. The bristle is seen on several cell types, and was later identified as a lattice of a single highly conserved protein4. This protein was named Clathrin, derived from the latin word for lattice. Clathrin is formed of light and heavy chains incorporated into a triskelion that assemble into closed hexagonal and pentagonal structures on the inner leaflet of the plasma membrane. Clathrin-mediated endocytosis has, since four decades ago, been recognized has an ubiquitous mechanism of plasma membrane uptake in cell types ranging from the frog presynaptic membrane5 to rat vas deferens6. 

	
\section{Clathrin mediated endocytosis in Mammalian vs Yeast}
		\subsection{Clathrin is required for mammalian CME}
That the Clathrin lattice is responsible for remodeling the plasma membrane and selecting cargo was speculated in the first papers that noted the “bristly” coat3,8.  In multicellular organisms like C.elegans, clathrin depleted by RNAi result in decreased endocytic uptake in oocytes and dead progeny9, in D.melanogaster, deletion of Clathrin heavy chain results in embryonic lethality10. In Hela cells, knock down of the heavy chain by RNAi results in decrease in endocytosis by 80\%11; essentially, endocytosis fails in the absence of clathrin. While early work that drove the field of endocytosis came from multicellular eukaryotes, genes involved in Clathrin-mediated endocytosis in yeast were found to be homologues of the mammalian machinery. The ease of genetic manipulation, establishment of the yeast genome, and relative simplicity of the endocytic pathway drove several discoveries in the yeast endocytic pathway that were later verified in mammalian cells. Early work in yeast revealed that clathrin was not necessary for endocytosis12. It became apparent that though the mammalian and yeast systems were quite similar mechanistically and most of the yeast endocytic proteins had mammalian homologues13, there are some significant differences. 

		\subsection{Actin is required for yeast CME}
CME is the only endocytic pathway in yeast. While the mammalian CME uptake is heavily dependent on clathrin, the yeast system relies on actin polymerization for endocytosis14. Cortical actin patches were first seen in S.cerevisae, that were later established as endocytic sites.  

	
		\subsection{Clathrin mediated endocytosis in yeast is modular}
	CME in yeast involves the recruitment, interaction, and disassembly of over fifty proteins. This process involves a flexible initiation phase that establishes endocytic sites and selects cargo15, then a very stereotypic sequence of events nucleates actin, organizes the actin network, invaginates a membrane tube, and finally severs the membrane to produce cargo-filled vesicles16. Work on yeast has shown that the initiation of endocytic sites is independent of the recruitment of any one protein15, and revealed how the the components17 and organization of the actin machinery allows the formation of a membrane invagination18–20. However, the mechanism of the final stage of membrane scission that leads to vesicle formation remains unclear.

		\subparagraph{The early phase}
		\subparagraph{The actin phase}
		\subparagraph{The scission phase}
		
		
	\section{Membrane scission in mammalian cells}
	\section{Membrane scission in yeast}
	
		
	\section{BAR domain proteins}
		\subsection{Mammalian vs yeast BAR proteins}
		\subsection{The Rvs complex}		
		
		
	
%\begin{itemize}
%	\item Is localization microscopy suitable to study endocytosis in budding yeast?
%\end{itemize}
